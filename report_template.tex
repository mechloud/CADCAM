\documentclass[12pt,twoside]{article}
%\usepackage[utf8x]{inputenc}
\usepackage[utf8]{inputenc}
\usepackage{amsmath}
\usepackage{graphicx}
\usepackage{wrapfig} %using for wrapping text around figures
\usepackage{tocloft}
\usepackage{float}
\usepackage{indentfirst}
\usepackage{siunitx}
\usepackage{array}
\usepackage{comment}
\usepackage{fancyhdr} % Used for headers and footers
% Used for table
\usepackage{xfrac}
\usepackage{pdfpages} % Used to include pdf attachments at the end of report
\usepackage{setspace}
\usepackage{subcaption}	% Used to add captions to subfigures
\usepackage{mwe} % also used for subfigures
\usepackage{enumitem} % used to control the vertical spacing between items in itemize or enumerate
\usepackage{lscape} % Used to make some pages landscape while leaving the majority in portrait
\usepackage{textcomp} % For registered, trademark and copyright symbols
\usepackage[hyphens]{url}
\urlstyle{same}
\usepackage[margin=1in]{geometry}
\usepackage[nottoc,numbib]{tocbibind} %https://tex.stackexchange.com/questions/8458/making-the-bibliography-appear-in-the-table-of-contents

\newcommand{\listappendicesname}{Appendices}
\newlistof{appendices}{apc}{\listappendicesname}
\newcommand{\appendices}[1]{\addcontentsline{apc}{appendices}{#1}}

\newcommand{\newappendix}[1]{\section*{#1}\appendices{#1}}

%%%%%%%%%%%%%%%%%%%%%%%%%%%%%%%%%%%%%%%%%%%%%%%%%%%%%%%%%%%%%%%%%%%%%%%%%%%
% HEADER & FOOTER SETTINGS
%%%%%%%%%%%%%%%%%%%%%%%%%%%%%%%%%%%%%%%%%%%%%%%%%%%%%%%%%%%%%%%%%%%%%%%%%%%

\pagestyle{fancy}
\fancyhf{}
\fancyhead[LE,RO]{MCG 4322 - BSAE2B}
\fancyhead[RE,LO]{Literature Review}
\fancyfoot[CE,CO]{\rightmark}
\fancyfoot[LE,RO]{\thepage}
\setlength{\headheight}{15pt} 
\renewcommand{\footrulewidth}{0.5pt} % sets the width of the line above the footer
\renewcommand{\sectionmark}[1]{\markright{#1}} % removes section number from footer

%%%%%%%%%%%%%%%%%%%%%%%%%%%%%%%%%%%%%%%%%%%%%%%%%%%%%%%%%%%%%%%%%%%%%%%%%%%
% HEADER & FOOTER STUFF
%%%%%%%%%%%%%%%%%%%%%%%%%%%%%%%%%%%%%%%%%%%%%%%%%%%%%%%%%%%%%%%%%%%%%%%%%%%

\pagestyle{fancy}
\fancyhf{}
\fancyhead[LE,RO]{MCG 4322 - BSAE2B}
\fancyhead[RE,LO]{Concepts Report}
\fancyfoot[CE,CO]{\rightmark}
\fancyfoot[LE,RO]{\thepage}
\setlength{\headheight}{15pt} 
\renewcommand{\footrulewidth}{0.5pt}
\renewcommand{\sectionmark}[1]{\markright{#1}} % removes section number from footer

%%%%%%%%%%%%%%%%%%%%%%%%%%%%%%%%%%%%%%%%%%%%%%%%%%%%%%%%%%%%%%%%%%%%%%%%%%%
%	GLOSSARY
%%%%%%%%%%%%%%%%%%%%%%%%%%%%%%%%%%%%%%%%%%%%%%%%%%%%%%%%%%%%%%%%%%%%%%%%%%%
\usepackage[acronym,toc,nomain,nonumberlist]{glossaries} % Used to make glossary & acronyms
\makeglossaries
% \newacronym{variable_name}{Acronym}{What the acronym stands for}
\newacronym{ALC}{ALC}{Aft Lateral Cross}
\newacronym{BLC}{BLC}{Overhead Lateral Cross}
\newacronym{CLC}{CLC}{Upper Lateral Cross}
\newacronym{DLC}{DLC}{Front Lateral Cross}
\newacronym{FAB}{FAB}{Fore-Aft Bracing}
\newacronym{FEA}{FEA}{Finite Element Analysis}
\newacronym{FBM}{FBM}{Front Bracing Member}
\newacronym{FBMlow}{FBM\textsubscript{LOW}}{Lower Front Bracing Member}
\newacronym{FBMup}{FBM\textsubscript{UP}}{Upper Front Bracing Member}
\newacronym{FLC}{FLC}{Front Lateral Cross}
\newacronym{HRS}{HRS}{Hot Rolled Steel}
\newacronym{LDB}{LDB}{Lateral Diagonal Bracing}
\newacronym{LFS}{LFS}{Lower Frame Side}
\newacronym{OCMF}{OCMF}{Orion Custom Metal Fabrication}
\newacronym{RHO}{RHO}{Roll Hoop Overhead}
\newacronym{RRH}{RRH}{Rear Roll Hoop}
\newacronym{SAE}{SAE}{Society of Automotive Engineers}
\newacronym{SIM}{SIM}{Side Impact Member}
\newacronym{USM}{USM}{Under Seat Member}


%%%%%%%%%%%%%%%%%%%%%%%%%%%%%%%%%%%%%%%%%%%%%%%%%%%%%%%%%%%%%%%%%%%%%%%%%%%
% DOCUMENT START
%%%%%%%%%%%%%%%%%%%%%%%%%%%%%%%%%%%%%%%%%%%%%%%%%%%%%%%%%%%%%%%%%%%%%%%%%%%

\begin{document}
	
	%%%%%%%%%%%%%%%%%%%%%%%%%%%%%%%%%%%%%%%%%%%%%%%%%%%%%%%%%%%%%%%%%%%%%%%%%%%
	% TITLE PAGE STARTS
	%%%%%%%%%%%%%%%%%%%%%%%%%%%%%%%%%%%%%%%%%%%%%%%%%%%%%%%%%%%%%%%%%%%%%%%%%%%
	\begin{titlepage}
		
		\newcommand{\HRule}{\rule{\linewidth}{0.5mm}} % Defines a new command for the horizontal lines, change thickness here
		
		\begin{center} % Center everything on the page
			
			%%%%%%%%%%%%%%%%%%%%%%%%%%%%%%%%%%%%%%%%%%%%%%%%%%%%%%%%%%%%%%%%%%%%%%%%%%%
			%	HEADING SECTIONS
			%%%%%%%%%%%%%%%%%%%%%%%%%%%%%%%%%%%%%%%%%%%%%%%%%%%%%%%%%%%%%%%%%%%%%%%%%%%
			
			\textsc{\LARGE University of Ottawa}\\[1.0cm] % Name of your university/college
			\textsc{\Large MCG 4322A}\\[0.5cm] % Major heading such as course name
			%\textsc{\large Minor Heading}\\[1.5cm] % Minor heading such as course title
			
			%%%%%%%%%%%%%%%%%%%%%%%%%%%%%%%%%%%%%%%%%%%%%%%%%%%%%%%%%%%%%%%%%%%%%%%%%%%
			%	TITLE SECTION
			%%%%%%%%%%%%%%%%%%%%%%%%%%%%%%%%%%%%%%%%%%%%%%%%%%%%%%%%%%%%%%%%%%%%%%%%%%%
			
			\HRule \\[0.4cm]
			{ \huge \bfseries Concepts Report}\\[0.4cm] % Title of your document
			\HRule \\[0.5cm]
			
			\begin{figure}[H]
				\centering
				\includegraphics[width=0.5\linewidth]{_team_picture1.jpg} % 0.7 times the width of the line
			\end{figure}
			
			%%%%%%%%%%%%%%%%%%%%%%%%%%%%%%%%%%%%%%%%%%%%%%%%%%%%%%%%%%%%%%%%%%%%%%%%%%%
			%	AUTHOR SECTION
			%%%%%%%%%%%%%%%%%%%%%%%%%%%%%%%%%%%%%%%%%%%%%%%%%%%%%%%%%%%%%%%%%%%%%%%%%%%
			
			\emph{Report written by:}\\[0.5cm]
			BSAE2B\\
			\textbf{Fluent Design}\\
			Jonathan \textsc{Charbonneau} 7199186\\
			Justin \textsc{McLeod} 7152573\\
			Elizabeth \textsc{Primeau} 7193663\\
			Alexandre \textsc{Vendette} 7301724\\[1cm]
			\emph{Presented to:}\\[0.5cm]
			Prof. Éric \textsc{Lanteigne}\\[1.5cm]
			
			
			%%%%%%%%%%%%%%%%%%%%%%%%%%%%%%%%%%%%%%%%%%%%%%%%%%%%%%%%%%%%%%%%%%%%%%%%%%%
			%	DATE SECTION
			%%%%%%%%%%%%%%%%%%%%%%%%%%%%%%%%%%%%%%%%%%%%%%%%%%%%%%%%%%%%%%%%%%%%%%%%%%%
			
			{\large September 28\textsuperscript{th}, 2017}\\
			
			\vfill % Fill the rest of the page with whitespace
			
		\end{center} % End centering
		
	\end{titlepage} % End the title page
	
	\renewcommand{\thepage}{\roman{page}} % Make page numbers roman numbers
	\newpage % Start on a new page
	\tableofcontents
	\thispagestyle{plain}
	\newpage
	\setstretch{1.3}
	\listoffigures
	\newpage
	\listoftables
    \newpage
    \thispagestyle{plain}
    \printglossary[type=\acronymtype]    
    \newpage
	
	% Turn the \thepage command back to arabic numbers for report body		
	\renewcommand{\thepage}{\arabic{page}}
	% Set the counter to 1
	\setcounter{page}{1}
		
	\section{Introduction}
    \subsection{Project Mandate}
    The objective of this project is to design, analyze and parametrize a functioning 4-wheeled, off-road vehicle in accordance with regulations stated in the Collegiate Design Series Baja \acrfull{SAE} Rules \cite{bajarules}. These regulations outline the acceptable vehicle components, build restrictions and prohibitions as well as specifications that ensure a minimum level of safety for the driver and everyone around the vehicle. 
    
    The Baja vehicles are designed and constructed by teams of university students. These teams can participate in multiple competitions across North America every year. During these competitions, the vehicles are put through different dynamic events to test the acceleration, hill climb, maneuverability and suspension. 
    
    As part of the project mandate, the design of the vehicle will be completed by two groups. One group must conceptualize all the component for the motion transfer and the other must design the remainder, such as the roll cage, suspension and steering. Both groups will work together as a team to design a complete and compliant Baja SAE vehicle.
    
    \subsection{Group Problem Scope}
    This group's objective is to design all non-motion transfer subsystems of the 4-wheeled, off-road vehicle in accordance to all regulations. The subsystems to be designed include the chassis, suspension, steering, body panels and anchor points. Additionally, all accessories and safety components such as the brake light, seat and seat tie points, and the mounting for the brake and gas pedals.
    
    The safety of the vehicle's operator is paramount. Therefore, all design selections must be made in favour of safety. The regulations must be strictly adhered to in order to produce a competitive and safe vehicle. The vehicle's chassis must be designed to eliminate all pinch and crush hazards presented by the suspension and steering systems, and separate the operator from the engine and fuel compartment. This report primarily deals with the design criteria and restrictions, concepts explored, cost assessment and decision making.
    
    \newpage\null\thispagestyle{empty}\newpage

    %%%%%%%%%%%%%%%%%%%%%%%%%%%%%%%%%%%%%%%%%%%%%%%%%%%%%%%%%%%%%%%%%%%%%%%%%%%%
    %	REFERENCES
    %%%%%%%%%%%%%%%%%%%%%%%%%%%%%%%%%%%%%%%%%%%%%%%%%%%%%%%%%%%%%%%%%%%%%%%%%%%% 

    \newpage
    \setstretch{1.0}
    \Urlmuskip=0mu plus 1mu\relax
    \bibliographystyle{IEEEtran}
    \bibliography{bibliography}
    \newpage
       	
    %%%%%%%%%%%%%%%%%%%%%%%%%%%%%%%%%%%%%%%%%%%%%%%%%%%%%%%%%%%%%%%%%%%%%%%%%%%%
    %	APPENDICES 
    %%%%%%%%%%%%%%%%%%%%%%%%%%%%%%%%%%%%%%%%%%%%%%%%%%%%%%%%%%%%%%%%%%%%%%%%%%%%

    % Appendices:
    % 	Data sheets (recommended if standard parts used)
    %   Decision Analysis Table
    % 	Additional Material

    \appendix
    \thispagestyle{plain}
    \addcontentsline{toc}{section}{Appendices}
    \renewcommand{\thesubsection}{\Alph{subsection}}
        		
    %%%%%%%%%%%%%%%%%%%%%%%%%%%%%%%%%%%%%%%%%%%%%%%%%%%%%%%%%%%%%%%%%%%%%%%%%%%%
    %	DATASHEETS
    %%%%%%%%%%%%%%%%%%%%%%%%%%%%%%%%%%%%%%%%%%%%%%%%%%%%%%%%%%%%%%%%%%%%%%%%%%%% 
    \section*{Appendices}
    \subsection{First Appendix}

    

\end{document}